\documentclass[a4paper, 11pt, twocolumn]{article}
\usepackage[utf8x]{inputenc}
\usepackage[IL2]{fontenc}
\usepackage[czech]{babel}
\usepackage[text={18cm,25cm}, top=2.5cm, left=1.5cm, includefoot]{geometry}
\usepackage{times}
\usepackage{amsmath, amsthm, amssymb}

\title{proj2}
\author{xvagne10}
\date{March 2021}

\theoremstyle{definition}
\newtheorem{definition}{Definice}
\newtheorem{sentence}{Věta}

\begin{document}

\begin{titlepage}

\begin{center}
\Huge
\textsc{Fakulta informačních technologií Vysoké~učení technické v Brně} \\
\vspace{\stretch{0.38}}

\LARGE
Typografie a publikování -- 2. projekt \\
Sazba dokumentů a matematických výrazů \\
\vspace{\stretch{0.618}}
\end{center}

\Large 2021 \hfill Dominik Vágner (xvagne10)

\end{titlepage}

\newpage{}

\section*{Úvod}
V této úloze si vyzkoušíme sazbu titulní strany, matematických vzorců, prostředí a dalších textových struktur obvyklých pro technicky zaměřené texty (například rovnice~(1) nebo Definice 1 na straně 1). Rovněž si
vyzkoušíme používání odkazů \verb$\ref$ a \verb$\pageref$.

Na titulní straně je využito sázení nadpisu podle optického středu s využitím zlatého řezu. Tento postup byl probírán na přednášce. Dále je použito odřádkování se zadanou relativní velikostí 0.4~em a 0.3~em.

V případě, že budete potřebovat vyjádřit matematickou konstrukci nebo symbol a nebude se Vám dařit jej nalézt v samotném \LaTeX u, doporučuji prostudovat možnosti balíku maker \AmS-\LaTeX.

\section{Matematický text}
Nejprve se podíváme na sázení matematických symbolů a~výrazů v plynulém textu včetně sazby definic a vět s využitím balíku \verb$amsthm$. Rovněž použijeme poznámku pod čarou s použitím příkazu \verb$\footnote$. Někdy je vhodné použít konstrukci \verb$\mbox{}$, která říká, že text nemá být zalomen.

\begin{definition} \label{def1}
Rozšířený zásobníkový automat \emph{(RZA)} \emph{je definován jako sedmice tvaru} $A=\left(Q, \Sigma, \Gamma, \delta, q_{0}, Z_{0}, F\right)$, \emph{kde:}
\end{definition} 

\begin{itemize}
    \item $Q$ \emph{je konečná množina} vnitřních (řídicích) stavů,
    \item $\Sigma$ \emph{je konečná} vstupní abeceda,
    \item $\Gamma$ \emph{je konečná} zásobníková abeceda,
    \item $\delta$ \emph{je} přechodová funkce $Q \times(\Sigma \cup \{\epsilon\}) \times \Gamma^{*} \rightarrow 2^{Q \times \Gamma^{*}}$,
    \item $q_{0} \in Q$ \emph{je} počáteční stav, $Z_{0} \in \Gamma$ \emph{je} startovací symbol zásobníku \emph{a} $F \subseteq Q$ \emph{je množina} koncových stavů.
\end{itemize}

Nechť $P=\left(Q, \Sigma, \Gamma, \delta, q_{0}, Z_{0}, F\right)$ je rozšířený zásobníkový automat. \emph{Konfigurací} nazveme trojici 
$(q, w, \alpha) \in Q \times \Sigma^{*} \times \Gamma^{*}$, kde $q$ je aktuální stav vnitřního řízení, $w$ je dosud nezpracovaná část vstupního řetězce a 
$\alpha = Z_{i_{1}} Z_{i_{2}} \ldots Z_{i_{k}}$ je obsah zásobníku\footnote[1]{$Z_{i_{1}}$ je vrchol zásobníku}.

\subsection{Podsekce obsahující větu a odkaz}
\begin{definition} \label{def2} 
Řetězec $w$ nad abecedou $\Sigma$ je přijat RZA \emph{A~jestliže} $(q_{0}, w, Z_{0}) \overset{*}{\underset{A}\vdash} 
(q_{F}, \epsilon, \gamma)$ \emph{pro nějaké} $\gamma \in \Gamma^{*}$ \emph{a} $q_{F} \in F$. \emph{Množinu} $L(A)=\{w~|~w~\emph{je přijat RZA A}\} \subseteq$
$\Sigma^{*}$~\emph{nazýváme} jazyk přijímaný RZA \emph{A}.
\end{definition}

Nyní si vyzkoušíme sazbu vět a důkazů opět s použitím balíku \verb$amsthm$.

\begin{sentence} 
\emph{Třída jazyků, které jsou přijímány ZA, odpovídá} bezkontextovým jazykům.
\end{sentence}

\begin{proof}
V důkaze vyjdeme z Definice \ref{def1} a \ref{def2}.
\end{proof}

\section{Rovnice a odkazy}
Složitější matematické formulace sázíme mimo plynulý text. Lze umístit několik výrazů na jeden řádek, ale pak je třeba tyto vhodně oddělit, například příkazem
\verb$\quad$.

\begin{equation*}
\sqrt[i]{x_{i}^{3}} \quad \text {kde } x_{i} \text { je } i \text {-té sudé číslo splňující} \quad x_{i}^{{{x_{i}^i}^2}+2} \leq y_{i}^{x_{i}^{4}}
\end{equation*}

V rovnici (1) jsou využity tři typy závorek s různou
explicitně definovanou velikostí.

\begin{eqnarray}
x & = & \left[\Big\{[a+b] * c\Big\}^{d} \oplus 2\right]^{3 / 2} \\
y & = & \lim_{x \rightarrow \infty} \frac{\frac{1}{\log _{10} x}}{\sin ^{2} x+\cos ^{2} x} \nonumber
\end{eqnarray}

V této větě vidíme, jak vypadá implicitní vysázení limity $\lim_{n \rightarrow \infty} f(n)$ v normálním odstavci textu. Podobně je to i s dalšími symboly jako
$\prod_{i=1}^{n} 2^{i} \text{ či } \bigcap_{A \in \mathcal{B}} A$. V případě vzorců $\lim\limits_{n \rightarrow \infty} f(n)$ a $\prod\limits_{i=1}^{n} 2^{i}$ 
jsme si vynutili méně úspornou sazbu příkazem \verb$\limits$.

\begin{eqnarray}
\int_{b}^{a} g(x)~\mathrm{d} x & = & -\int\limits_{a}^{b} f(x)~\mathrm{d} x
\end{eqnarray}

\section{Matice}
Pro sázení matic se velmi často používá prostředí \verb$array$ a závorky (\verb$\left$, \verb$\right$).

\begin{equation*}
\left(\begin{array}{ccc}
a - b & \widehat{\xi+\omega} & \pi \\
\vec{\mathbf{a}} & \overleftrightarrow{A C} & \hat{\beta}
\end{array}\right) = 1 \Longleftrightarrow \mathcal{Q} = \mathbb{R}
\end{equation*}

\begin{equation*}
\mathbf{A}=\left\|\begin{array}{cccc}
a_{11} & a_{12} & \ldots & a_{1 n} \\
a_{21} & a_{22} & \ldots & a_{2 n} \\
\vdots & \vdots & \ddots & \vdots \\
a_{m 1} & a_{m 2} & \ldots & a_{m n}
\end{array}\right\|=\left|\begin{array}{cc}
t & u \\
v & w
\end{array}\right|=t w-u v
\end{equation*}

Prostředí \verb$array$ lze úspěšně využít i jinde.

\begin{equation*}
\binom{n}{k} = \left\{\begin{array}{cl}
0 & \text { pro } k<0 \text { nebo } k>n \\
\frac{n !}{k !(n-k) !} & \text { pro } 0 \leq k \leq n
\end{array}\right.
\end{equation*}

\end{document}