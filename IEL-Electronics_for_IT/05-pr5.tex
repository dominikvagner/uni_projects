\section{Příklad 5}
% Jako parametr zadejte skupinu (A-H)
\patyZadani{G}

\vspace{1cm}
\large{\textbf{Rešení (sestavení diferenciální rovnice pro chování daného obvodu a výpočet analytického rešení):}}
\vspace{0.5cm}

%%% Krok 1
\begin{center}
\textbf{Krok 1} - Sestavíme si rovnice pro $i_L$ a $i'_L$. \\
\end{center}
\vspace{-0.5cm}

\begin{gather*}
i_L = \frac{U_R}{R}\\\\
\newline
\newline
i'_L = \frac{U_L}{L}\\\\
\end{gather*}

%%% Krok 2
\begin{center}
\textbf{Krok 2} - Vytvoříme rovnici podle II. Kirchhoffového zákona a upravíme ji. \\
\end{center}
\vspace{-0.5cm}

\begin{gather*}
U_R + U_L - U = 0\\\\
\newline
\newline
U = R \times i_L + L \times i'_L\\\\
\end{gather*}

\newpage

%%% Krok 3
\begin{center}
\textbf{Krok 3} - Vypočítáme $\lambda$ z očekávaného tvaru rovnice. \\
\end{center}
\vspace{-0.5cm}

\begin{gather*}
\text{Očekávaný tvar:}\ i_L(t) = k(t) \times e^{\lambda \times t} \\\\
\newline
\newline
\lambda = ? \\\\
\newline
\newline
\lambda \times L + R = 0\\\\
\newline
\newline
\lambda = -\frac{R}{L}\\\\
\end{gather*}

%%% Krok 4
\begin{center}
\textbf{Krok 4} - Dosadíme vypočítanou $\lambda$ do očekávaného tvaru a derivujeme. \\
\end{center}
\vspace{-0.5cm}

\begin{gather*}
i_L(t) = k(t) \times e^{-\frac{R}{L}t} \\\\
\newline
\newline
i'_L(t) = k'(t) \times e^{-\frac{R}{L}t} + k(t) \times (-\frac{R}{L}) \times e^{-\frac{R}{L}t} \\\\
\newline
\newline
\end{gather*}


%%% Krok 5
\begin{center}
\textbf{Krok 5} - Dosadíme do rovnice ze druhého kroku a upravíme. \\
\end{center}
\vspace{-0.5cm}

\begin{gather*}
R \times k(t) \times e^{-\frac{R}{L}t} + L \times (k'(t) \times e^{-\frac{R}{L}t} + k(t) \times (-\frac{R}{L}) \times e^{-\frac{R}{L}t}) = U \\\\
\newline
\newline
L \times k'(t) \times e^{-\frac{R}{L}t} = U \\\\
\newline
\newline
k'(t) = \frac{U}{L} \times e^{\frac{R}{L}t} \ / \int \\\\
\newline
\newline
k(t) = \frac{\frac{U}{L}}{\frac{R}{L}} \times e^{\frac{R}{L}t} + K \\\\
\newline
\newline
k(t) = \frac{U}{R} \times e^{\frac{R}{L}t} + K\\\\
\newline
\newline
\end{gather*}

\newpage

%%% Krok 6
\begin{center}
\textbf{Krok 6} - Dosadíme k(t) do očekávané rovnice. \\
\end{center}
\vspace{-0.5cm}

\begin{gather*}
i_L =  (\frac{U}{R} \times e^{\frac{R}{L}t} + K) * e^{-\frac{R}{L}t} \\\\
\newline
\newline
i_L = \frac{U}{R} + K \times e^{-\frac{R}{L}t}\\\\
\newline
\newline
\end{gather*}

%%% Krok 7
\begin{center}
\textbf{Krok 7} - Vyjádříme si K podle podmínky $i_L(t) = 8 A$ (kde $t = 0$) a vytvoříme analaytické řešení. \\
\end{center}
\vspace{-0.5cm}

\begin{gather*}
i_L(0) = \frac{U}{R} + K \times e^{-\frac{R}{L} \times 0} \\\\
\newline
\newline
K = i_L(0) - \frac{U}{R} \\\\
\newline
\newline
\textbf{Analitické řešení:} \\\\
\newline
\newline
i_L = \frac{U}{R} + (i_L(0) - \frac{U}{R}) \times e^{-\frac{R}{L}t} \\\\
\end{gather*}

%%% Krok 8
\begin{center}
\textbf{Krok 8} - Kontrolu provedeme dosazením hodnot do analitického řešení s hodntou $t = 0$. \\
\end{center}
\vspace{-0.5cm}

\begin{gather*}
i_L = \frac{U}{R} + (i_L(0) - \frac{U}{R}) \times e^{-\frac{R}{L}t} \\\\
\newline
\newline
8 = \frac{20}{25} + (8-\frac{20}{25}) * e^0 \\\\
\newline
\newline
8\ =\ 8 \\\\
\end{gather*}